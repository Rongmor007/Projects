% Проверка тока обтекания пиросредств

% получение всех шагов главной последовательности
%% set substeps = e.get_substeps(step, 'NumericLimitTest', 'Main')

% Получение статуса шага
%% set status = h.translate_en_to_ru(e.get_step_status(step))

% Заголовок 
\multicolumn{1}{|C{\VAR{w4}}|}{\VAR{current_step_name[0]}} & \multicolumn{4}{C{\VAR{w2_7}}|}{\VAR{current_step_name[1]}} &  \multicolumn{1}{C{\VAR{w3}}|}{\textbf{\VAR{status}}} \\

% prec - сколько знаков после запятой оставить
%% set prec = 4
%% set len_of_rows = [2]
%% set len = len_of_rows|length

% Списки для заполнения таблицы
%% set Max_amp = []
%% set lo_limit = []
%% set Status_substep = []

%% for step in substeps
%% set subs = [step]
% Лимиты и текущие значения
%% set limits = e.get_numeric_limits(subs, (e.get_step_name(step)))
% Высчитываем максимальный лимит
%% set dump = lo_limit.append(limits[0]['lo'])
% Высчитываем максимальный ток
%% set dump = Max_amp.append(limits[0]['result'])
%% set dump = Status_substep.append(e.get_step_status(step))
%% endfor
% Максимальное значение 
%% set max_amper = Max_amp | max

% Проверка на успешность тестов
%% set Status_substep = h.status_of_substeps(Status_substep)
\hhline{~|--|--|--|-}
& \multicolumn{1}{C{\VAR{w2_3}}|}{\multirow{3}{50mm}{\centering Максимальный ток обтекания ПС}} & \multicolumn{2}{C{\VAR{w6_7}}|}{Измерено (мА)} & \multicolumn{1}{C{\VAR{w3_6}}|}{Допустимое значение (мА)} & \tabularnewline \hhline{~|~~|--|--|-}

& \multicolumn{1}{C{\VAR{w2_3}}|}{} & \multicolumn{2}{C{\VAR{w6_7}}|}{\VAR{max_amper}} &  \multicolumn{1}{C{\VAR{w3_6}}|}{<= \VAR{lo_limit[0]}} & \multicolumn{1}{C{25.06mm}|}{\textbf{\VAR{Status_substep}}} 
