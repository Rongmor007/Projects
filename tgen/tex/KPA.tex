\documentclass[12pt,a4paper]{report}
% Русский язык
\usepackage[english, russian]{babel}
\usepackage{fontspec}	% пакет для компилятора XeLaTeX и LuaLaTeX

\usepackage[left=1.00cm, right=1.00cm, top=2.00cm, bottom=2.00cm]{geometry} % Макет документа
\usepackage{etex}
\usepackage{array} % для расширения возможностей построения таблиц
\usepackage{longtable} % для построения таблиц
\usepackage{multirow} % для рисования таблиц (объединение строк по вертикали)
\usepackage{hhline}	% для рисования горизонтальных линий таблиц
\usepackage{setspace} % пробел между линиями
\usepackage{longtable} % длинные таблицы

% Математика 
\usepackage{mathtext}
\usepackage{makeidx}

% Графики
% \usepackage{tikz}
% \usepackage{tikzexternal}
\usepackage{pgfplots} % основной

% Заголовок таблицы
\usepackage{caption}
\captionsetup{justification = raggedright, singlelinecheck = false}

% включение шрифта Times New Roman
\renewcommand{\rmdefault}{ftm} 
\setmainfont{Times New Roman} % шрифт Times New Roman при подключении пакета fontspec

\hyphenpenalty=10000 % запрет на автоматический перенос слов с расширением текущей строки

% объявление нового типа колонки
\newcolumntype{C}[1]{>{\centering}m{#1}} % добавления возможности форматирования текста по центру в таблицах
\newcommand{\VAR}[1]{} % объявление для латеха команды содержащей переменную
\newcommand{\BLOCK}[1]{} % объявление для латеха команды содержащей блок

\begin{document}
	% объявление количества табличных строк, размещаемых на одной странице
	% 48 строк было раньше, но они иногда не вмещались
	%% set total_count_of_rows = 47
	% объявление количества символов в одной строке в шести объединённых центральных клетках таблицы
	%% set count_of_symbols_in_central_row = 60
	% объявление количества символов в одной строке в двух объединённых столбцах под номерами 2 и 3 (нумерация с 1)
	%% set count_of_symbols_in_2_3_columns = 26
	% объявление количества символов во втором столбце
	%% set count_of_symbols_in_second_column = 13
	% ширина для объединённых колонок 2 и 3
	%% set w2_3 = '57.375mm'
	%% set w2_5 = '96.245mm'
	%% set w3_5 = '63.685mm'
	%% set w4_5 = '34.875mm'
	%% set w6_7 = '37.25mm'
	%% set w4_7 = '76.125mm'
	%% set w2_6 = '141mm'
	%% set w2_7 = '137.75mm'
	%% set w3_7 = '105.44mm'
	%% set w3 = '28.06mm'
	%% set w4 = '10.185mm'
	%% set w3_6 = '35mm'
	%% set w2 = '16.5mm'
	%% set w2_4 = '26.5mm'
	%% set w4_1 = '25mm'
	%% set w5 = w4
	% объявление номера текущей строки
	%% set ns = namespace(number_of_rows = 0, one_header='', count_of_share_rows = 0, global_counter = 0)
	
	% считывание данных из нового файла
	%% set dump = e.set_node(list_of_reports[0])
	%% set number_of_test = e.get_uut_serial_number()
	%% set device = 'Прибор'
	%% set date = h.exist_result(h.clean_date(e.get_start_time()))
	%% set defaults_header = h.get_header(0)
%	\begin{spacing}{1.3}
%		\fontsize{14pt}{\baselineskip}\selectfont
%		\pagestyle{empty} % страница без номера
%		\setcounter{page}{1} % счетчик страницы
%		\begin{center}
%			ПРОТОКОЛ \\
%			самопроверки
%			КПА БИОС \\
%			ЖКНЮ.***
%			\vspace{2em plus .6em minus .5em}\\
%			
%			Листов
%			\clearpage
%		\end{center}
%	\end{spacing}
	\begin{center}
		\begin{spacing}{1.3}
			\setcounter{page}{1}
			\fontsize{14pt}{\baselineskip}\selectfont
			%% set words = defaults_header['head1'].split()
			%% include('HeaderDWNT.tex') 
			\textbf{\VAR{ns.one_header}}
			
			%% set words = defaults_header['gk'].split()
			%% include('HeaderDWNT.tex') 
			\textbf{\VAR{ns.one_header}}
						
			%% set words = defaults_header['head2'].split()
			%% include('HeaderDWNT.tex')
			\textbf{\VAR{ns.one_header}}
			
			%% set words = defaults_header['head3'].split()
			%% include('HeaderDWNT.tex')
			\textbf{\VAR{ns.one_header}}
			
			%% set words = defaults_header['date'].split()
			%% include('HeaderDWNT.tex')
			\textbf{\VAR{ns.one_header}}

		\end{spacing}
	\end{center}

	% 20 ГК
	\begin{center}
	%% if list_of_gk[0] == '20GK'
	\label{:1}
	%% set steps = e.get_steps('trc:TestResults/tr:ResultSet', 'SequenceCall', 'Main')
	%% set dump = e.sort_list_of_steps(steps)
	%% set ns.number_of_rows = 8 + 0
	% set head_of_tabular = '\\begin{longtable}{|C{11mm}|C{28mm}|C{25mm}|*{2}{C{15mm}|}C{14mm}|C{19mm}|C{29mm}|} \\hline\n№ & \\multicolumn{6}{C{' + w2_7 + '}|}{Проверка} & Результат \\tabularnewline \\hline'
	
	%% for step in steps:
	%% set step_status = h.exist_result(e.get_step_status(step))
	%% set step_name = h.exist_result(e.get_step_name(step))
	%% set number = h.number_of_step(step_name)
	
	% Разделения имени шага на его номер и название
	%% set current_step_name = h.split_name(step_name)
	
	%% if number != '19' and number != '12' and step_status != 'Пропущено' and step_status != 'Не запускалась' and step_status != 'Прервана'
	\begin{longtable}{|C{11mm}|C{28mm}|C{25mm}|*{2}{C{15mm}|}C{14mm}|C{19mm}|C{29mm}|}
		\caption{} \\ % Делаемглавную шапку таблицы. Она будет одна и не должна повторяться
		\hline
		\multicolumn{1}{|C{\VAR{w4}}|}{№} & \multicolumn{4}{C{\VAR{w2_7}}|}{Проверка} & \multicolumn{1}{C{\VAR{w3}}|}{Результат} \\
		\hline
		\endfirsthead
		\multicolumn{6}{l} % объединение колонок
		{\tablename\ \thetable\ -- \textit{Перенос с предыдущей страницы}} \\
		\hline
		\multicolumn{1}{|C{\VAR{w4}}|}{№} & \multicolumn{4}{C{\VAR{w2_7}}|}{Проверка} & \multicolumn{1}{C{\VAR{w3}}|}{Результат} \\ 
		\hline % Название колонок таблицы на каждой новой странице
		\endhead
		\hline % Линия в конце страницы под таблицей
		\multicolumn{6}{r}{\textit{Продолжение на следующей странице}} \\ % Подпись таблицы (справа в данном случае, т.к. {r}) в конце каждой страницы 
		\endfoot % окончание каждой страницы кроме последней
		\hline
		\endlastfoot % окончание последней страницы
		%% include ('bios' + number + '_20gk.tex')
	\end{longtable}
	%% endif
	
	%% if (number == '19' or number == '12') and step_status != 'Пропущено' and step_status != 'Не запускалась' and step_status != 'Прервана'
		%% include ('bios' + number  + '_20gk' + '.tex')
	%% endif
	%% endfor
	\newpage
	\noindent
	%% endif
	\end{center}

	% 50 ГК
	\begin{center}
	%% if list_of_gk[0] == '50GK'
	\label{:1}
	%% set steps = e.get_steps('trc:TestResults/tr:ResultSet', 'SequenceCall', 'Main')
	%% set dump = e.sort_list_of_steps(steps)
	%% set ns.number_of_rows = 8 + 0
	% set head_of_tabular = '\\begin{longtable}{|C{11mm}|C{28mm}|C{25mm}|*{2}{C{15mm}|}C{14mm}|C{19mm}|C{29mm}|} \\hline\n№ & \\multicolumn{6}{C{' + w2_7 + '}|}{Проверка} & Результат \\tabularnewline \\hline'
	
	%% for step in steps:
	%% set step_status = h.exist_result(e.get_step_status(step))
	%% set step_name = h.exist_result(e.get_step_name(step))
	%% set number = h.number_of_step(step_name)
	
	% Разделения имени шага на его номер и название
	%% set current_step_name = h.split_name(step_name)
	
	%% if number != '19' and number != '12' and step_status != 'Пропущено' and step_status != 'Не запускалась' and step_status != 'Прервана'
	\begin{longtable}{|C{11mm}|C{28mm}|C{25mm}|*{2}{C{15mm}|}C{14mm}|C{19mm}|C{29mm}|}
		\caption{} \\ % Делаемглавную шапку таблицы. Она будет одна и не должна повторяться
		\hline
		\multicolumn{1}{|C{\VAR{w4}}|}{№} & \multicolumn{4}{C{\VAR{w2_7}}|}{Проверка} & \multicolumn{1}{C{\VAR{w3}}|}{Результат} \\
		\hline
		\endfirsthead
		\multicolumn{6}{l} % объединение колонок
		{\tablename\ \thetable\ -- \textit{Перенос с предыдущей страницы}} \\
		\hline
		\multicolumn{1}{|C{\VAR{w4}}|}{№} & \multicolumn{4}{C{\VAR{w2_7}}|}{Проверка} & \multicolumn{1}{C{\VAR{w3}}|}{Результат} \\ 
		\hline % Название колонок таблицы на каждой новой странице
		\endhead
		\hline % Линия в конце страницы под таблицей
		\multicolumn{6}{r}{\textit{Продолжение на следующей странице}} \\ % Подпись таблицы (справа в данном случае, т.к. {r}) в конце каждой страницы 
		\endfoot % окончание каждой страницы кроме последней
		\hline
		\endlastfoot % окончание последней страницы
		%% include ('bios' + number + '_50gk.tex')
	\end{longtable}
	%% endif
	
	%% if (number == '19' or number == '12') and step_status != 'Пропущено' and step_status != 'Не запускалась' and step_status != 'Прервана'
		%% include ('bios' + number  + '_50gk' + '.tex')
	%% endif
	%% endfor
	\newpage
	\noindent
	%% endif
	\end{center}
	
%	\begin{center}	
%	\begin{minipage}{0.35\textwidth}
%		\begin{center}
%			
%			(Начальник ЭП)
%		\end{center}
%	\end{minipage}\hfill
%	\begin{minipage}{0.15\textwidth}
%		\begin{center}
%			\rule[-1.8mm]{2.5cm}{.1mm}
%
%			подпись
%		\end{center}
%	\end{minipage}\hfill
%	\begin{minipage}{0.35\textwidth}
%		\begin{center}
%			\rule[-1.8mm]{5cm}{.1mm}
%			
%			инициалы, фамилия
%		\end{center}
%	\end{minipage}\hfill
%	\begin{minipage}{0.15\textwidth}
%		\begin{center}
%			\rule[-1.8mm]{2.5cm}{.1mm}
%			
%			дата
%		\end{center}
%	\end{minipage}\vspace{3em plus 1.1em minus .8em}
%	
%	\noindent
%	\begin{minipage}{0.35\textwidth}
%		\begin{center}
%			 В.П.
%			
%			(Начальник лаборатории)
%		\end{center}
%	\end{minipage}\hfill
%	\begin{minipage}{0.15\textwidth}
%		\begin{center}
%			\rule[-1.8mm]{2.5cm}{.1mm}
%			
%			подпись
%		\end{center}
%	\end{minipage}\hfill
%	\begin{minipage}{0.35\textwidth}
%		\begin{center}
%			\rule[-1.8mm]{5cm}{.1mm}
%			
%			инициалы, фамилия
%		\end{center}
%	\end{minipage}\hfill
%	\begin{minipage}{0.15\textwidth}
%		\begin{center}
%			\rule[-1.8mm]{2.5cm}{.1mm}
%			
%			дата
%		\end{center}
%	\end{minipage}
%	\label{p:1}
%	\pagestyle{empty}	
%	\end{center}
\end{document}